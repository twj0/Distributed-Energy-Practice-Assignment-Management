\documentclass[a4paper,12pt]{article}
\usepackage[utf8]{inputenc}
\usepackage{ctex} % 支持中文
\usepackage{amsmath,amssymb} % 数学公式支持
\usepackage{geometry} % 页面布局
\usepackage{graphicx} % 插入图片
\usepackage{caption} % 图片和表格标题
\usepackage{booktabs} % 表格美化
\usepackage{pythonhighlight}
\geometry{a4paper, margin=1in}

% 设置页面标题和作者信息
\title{基于科学增强型优化器的冷热电联供系统优化平台研究}
\author{唐玮嘉 \\ 学号: 2023428020130}
\date{2025年5月6日}

\begin{document}

\maketitle

% 摘要
\begin{abstract}
冷热电联供系统(CCHP)通过同时提供电力、热能和冷能,实现能源的高效综合利用。然而,其较高的初投资和运行成本要求对其进行经济优化。本文基于教材内容,建立了CCHP系统的热经济性模型,并开发了科学增强型优化器(v5.0)以求解优化问题。通过实验验证了优化器的有效性,展示了优化过程和结果的可视化分析。研究表明,该优化平台能够显著提升系统的热经济性,为实际工程应用提供了理论支持。
\end{abstract}

% 正文部分
\section{引言}
冷热电联供系统(CCHP)是一种先进的能源供应方式,能够同时满足电力、热能和冷能需求,从而显著提高能源综合利用效率。然而,由于系统初投资和运行成本较高,其热经济性受到挑战。工程实践表明,只有通过充分回收利用余热,CCHP系统才能体现出节能和经济优势。因此,对CCHP系统进行经济优化成为研究的关键问题。本文旨在构建一个科学严谨的优化平台,通过数学建模和优化算法,探索CCHP系统的最优运行策略。

\section{系统模型}
根据教材内容,本文建立了微型CCHP系统的热经济性模型,包括目标函数和约束条件,以量化系统的综合成本。

\subsection{目标函数}
目标函数表示系统的综合成本,旨在最小化运行和投资成本的加权和。本文测试了以下三种形式的目标函数,其中$x_1$和$x_2$分别代表发电效率系数和余热回收系数:
\begin{itemize}
    \item 模式1:$f(x) = 10(x_1 - 1)^2 + (x_2 + 1)^4$
    \item 模式2:$f(x) = 100(x_1^2 - x_2)^2 + (x_1 - 1)^2$
    \item 模式3:$f(x) = 100(x_1^2 - 3x_2)^2 + (x_1 - 1)^2$
\end{itemize}
这些函数分别模拟了高电负荷、热电平衡和余热优先的运行场景,具有非线性特征,适合测试优化算法的性能。

\subsection{约束条件}
考虑到物理和操作限制,参数需满足以下可行域约束:
\[
0 \leq x_1 \leq 1, \quad 0 \leq x_2 \leq 1
\]
上述约束确保发电效率和余热回收系数在实际可行范围内。

\section{优化方法}
为求解上述优化问题,本文开发了科学增强型优化器(v5.0),集成了多种高级优化技术,以确保高效和稳定的收敛。

\subsection{优化器核心特性}
优化器包含以下关键技术:
\begin{itemize}
    \item \textbf{精确投影梯度法}:确保搜索方向始终在可行域内。
    \item \textbf{自适应步长与回溯线搜索}:动态调整步长以加速收敛。
    \item \textbf{动态动量与二次插值}:增强搜索稳定性。
    \item \textbf{边界弹性缓冲机制}:避免参数在边界处震荡。
    \item \textbf{多维收敛稳定性诊断}:综合评估优化过程的收敛性。
\end{itemize}

\subsection{算法细节}
优化器基于梯度下降法,结合动量法和自适应学习率,具体步骤如下:
\begin{enumerate}
    \item 计算目标函数的梯度 $\nabla f(x)$。
    \item 将梯度投影到可行域内,确保方向可行性。
    \item 在边界附近动态增强梯度,避免陷入局部解。
    \item 根据梯度范数和优化进展调整学习率。
    \item 使用回溯线搜索确定最优步长。
    \item 更新参数并投影至可行域。
    \item 通过多维指标检测收敛性。
\end{enumerate}
上述步骤通过代码实现,保证了算法的科学性和严谨性。

\section{实验与结果}
为验证优化器的性能,本文对三种目标函数进行了优化实验,并分析了结果。

\subsection{实验设置}
实验采用以下配置:
\begin{itemize}
    \item 运行模式:mode1(高电负荷)、mode2(热电平衡)、mode3(余热优先)。
    \item 初始参数:根据模式选择合理初值(如mode2选用$[0.3, 0.3]$)。
    \item 优化参数:最大迭代次数1000,收敛阈值$10^{-5}$。
\end{itemize}

\subsection{结果分析}
\begin{figure}
    \centering
    \includegraphics[width=0.8\textwidth]{fig/cchp_optimization_mode2-0.png}
    \caption{模型2默认的优化轨迹}
    \label{fig:trajectory}
\end{figure}

实验结果表明,优化器在所有模式下均快速收敛。以mode2为例,默认的优化轨迹和成本收敛曲线如下所述:


\begin{figure}
    \centering
    \includegraphics[width=0.8\textwidth]{fig/cchp_optimization_mode2-1.png}
    \caption{模型2自定义的优化成本收敛曲线}
    \label{fig:cost}
\end{figure}

自定义的优化过程,输出结果如下:\\
优化结果报告\\
运行模式:       mode2\\
初始参数:       [0.5, 0.5]\\
迭代次数:       1000\\
计算耗时:       0.195秒\\
最终发电效率:   0.601524\\
最终余热回收:   0.369493\\
最终成本值:     1.646527e-01\\

三种模式的最终成本均达到预期优化目标,验证了优化器的有效性和鲁棒性。

\section{结论}
本文提出了一种基于科学增强型优化器的CCHP系统优化平台,通过建立热经济性模型并结合高效优化算法,成功实现了系统的经济优化。实验结果表明,该平台能够在不同运行模式下显著降低综合成本,具有较强的实用价值。未来研究可进一步扩展至更复杂的系统模型,并探索其他高级优化算法,以提升平台的性能和适用性。



\section{附录}
\begin{python}
"""
冷热电联供系统优化平台 - 科学增强版 (v5.0)
2023428020130
唐玮嘉
2025 0506
核心改进:
1. 精确投影梯度计算与边界安全机制
2. 自适应学习率与线搜索技术
3. 多阶段动量调整算法
4. 强化约束处理与边界弹性技术
5. 稳定收敛保障与诊断系统
"""

import numpy as np
import matplotlib.pyplot as plt
from scipy.optimize import minimize, line_search
import time
import os
import copy

# 可视化配置
plt.rcParams['font.sans-serif'] = ['Microsoft YaHei']
plt.rcParams['axes.unicode_minus'] = False

class ScientificCCHPOptimizer:
    """
    科学增强型冷热电联供系统优化器(v5.0)
    
    核心特性:
    - 精确投影梯度法边界处理
    - 自适应步长与回溯线搜索
    - 动态动量与二次插值技术
    - 边界弹性缓冲机制
    - 多维收敛稳定性诊断
    """
    
    # 参数可行域 [发电效率, 余热回收] ∈ [0,1]^2
    PARAM_BOUNDS = np.array([[0.0, 1.0], [0.0, 1.0]])
    
    def __init__(self, mode='mode2', config=None):
        """
        初始化科学增强优化器
        :param mode: 运行模式 (mode1/mode2/mode3)
        :param config: 配置字典
        """
        self.mode = mode
        self.config = {
            'max_iters': 1000,         # 最大迭代次数
            'epsilon': 1e-5,           # 收敛阈值
            'base_lr': 0.1,            # 基础学习率
            'momentum': 0.9,           # 动量系数
            'buffer_width': 0.05,      # 边界缓冲层宽度(参数范围的5%)
            'gradient_boost': 2.0,     # 边界附近梯度增强系数
            'line_search': True,       # 启用线搜索
            'adaptive_momentum': True, # 动态动量调整
            'momentum_decay': 0.95,    # 动量衰减系数
            'wall_rebound': 0.5,       # 边界反弹系数
            **(config or {})
        }
        
        # 优化状态变量
        self.params = np.array([0.0, 0.0])  # 当前参数 [x1, x2]
        self.velocity = np.zeros(2)         # 动量项
        self.best_params = None             # 历史最优参数
        self.best_cost = float('inf')       # 历史最优成本
        self.last_decay_iter = 0            # 上次动量衰减迭代
        self.stagnation_counter = 0         # 收敛停滞计数
        
        self.history = {                    # 优化过程记录
            'params': [],                   # 参数轨迹
            'costs': [],                    # 成本值
            'gradients': [],                # 梯度记录
            'violations': [],               # 边界违规
            'deltas': [],                   # 参数变化量
            'learning_rates': [],           # 学习率记录
            'momentum_factors': []          # 动量因子记录
        }

        # 根据模式初始化目标函数
        self._init_objective()
        
        # 初始化问题维度
        self.n_dim = len(self.params)
        
        # 初始化智能收敛检测器
        self.converge_history = np.zeros((5, 10))  # 5种收敛指标的最近10次状态

    def _init_objective(self):
        """初始化目标函数与解析梯度"""
        # 目标函数定义
        if self.mode == 'mode1':
            self.objective = lambda x: 10*(x[0]-1)**2 + (x[1]+1)**4
            self.gradient = lambda x: np.array([20*(x[0]-1), 4*(x[1]+1)**3])
        elif self.mode == 'mode2':
            self.objective = lambda x: 100*(x[0]**2 - x[1])**2 + (x[0]-1)**2
            self.gradient = lambda x: np.array([
                400*x[0]*(x[0]**2 - x[1]) + 2*(x[0]-1),
                -200*(x[0]**2 - x[1])
            ])
        elif self.mode == 'mode3':
            self.objective = lambda x: 100*(x[0]**2 - 3*x[1])**2 + (x[0]-1)**2
            self.gradient = lambda x: np.array([
                400*x[0]*(x[0]**2 - 3*x[1]) + 2*(x[0]-1),
                -600*(x[0]**2 - 3*x[1])
            ])
        else:
            raise ValueError("未知的运行模式")

    def _project_params(self, params):
        """将参数精确投影到可行域内"""
        return np.clip(params, self.PARAM_BOUNDS[:,0], self.PARAM_BOUNDS[:,1])

    def _project_gradient(self, params, raw_grad):
        """
        精确投影梯度 - 确保搜索方向的可行性
        :param params: 当前参数
        :param raw_grad: 原始梯度
        :return: 投影后的可行梯度
        """
        projected_grad = raw_grad.copy()
        
        # 对每个维度执行边界检查
        for i in range(self.n_dim):
            lb, ub = self.PARAM_BOUNDS[i]
            
            # 下边界检测 (参数接近下边界)
            if params[i] <= lb + 1e-6:
                # 只有当梯度指向可行域内部时才保留
                projected_grad[i] = max(0, projected_grad[i])
            
            # 上边界检测 (参数接近上边界)
            if params[i] >= ub - 1e-6:
                # 只有当梯度指向可行域内部时才保留
                projected_grad[i] = min(0, projected_grad[i])
                
        # 确保投影后的梯度不全为零(避免完全卡住)
        if np.all(np.abs(projected_grad) < 1e-10) and np.any(np.abs(raw_grad) > 1e-6):
            # 退化为最陡下降方向的小扰动
            epsilon = 1e-6
            perturbation = -epsilon * raw_grad / (np.linalg.norm(raw_grad) + 1e-12)
            
            # 检查扰动是否会导致越界
            proposed_params = params + perturbation
            if np.all((proposed_params >= self.PARAM_BOUNDS[:,0]) & 
                      (proposed_params <= self.PARAM_BOUNDS[:,1])):
                projected_grad = perturbation
                
        return projected_grad

    def _dynamic_gradient_adjust(self, params, grad):
        """
        动态梯度调整(边界缓冲层增强)
        :param params: 当前参数
        :param grad: 原始梯度
        :return: 调整后的梯度
        """
        adjusted_grad = grad.copy()
        buffer = self.config['buffer_width']
        
        # 对每个参数进行边界检测
        for i in range(self.n_dim):
            lb, ub = self.PARAM_BOUNDS[i]
            
            # 下边界缓冲增强 - 仅当梯度方向正确时增强
            if (params[i] - lb) < buffer and adjusted_grad[i] > 0:
                scale = 1 + (buffer - (params[i] - lb))/buffer * self.config['gradient_boost']
                adjusted_grad[i] *= scale
                
            # 上边界缓冲增强 - 仅当梯度方向正确时增强
            if (ub - params[i]) < buffer and adjusted_grad[i] < 0:
                scale = 1 + (buffer - (ub - params[i]))/buffer * self.config['gradient_boost']
                adjusted_grad[i] *= scale
                
        return adjusted_grad

    def _adaptive_learning_rate(self, iter_num, grad_norm, recent_progress):
        """
        增强型自适应学习率
        :param iter_num: 当前迭代次数
        :param grad_norm: 当前梯度范数
        :param recent_progress: 最近的优化进展
        :return: 调整后的学习率
        """
        base_lr = self.config['base_lr']
        
        # 基于梯度范数的缩放
        if grad_norm > 10.0:
            lr_scale = 0.1  # 大梯度时保守更新
        elif grad_norm > 1.0:
            lr_scale = 0.5  # 中等梯度适中更新
        elif grad_norm < 0.01:
            lr_scale = 2.0  # 小梯度加速收敛
        else:
            lr_scale = 1.0
            
        # 基于收敛进展的调整
        if recent_progress < 1e-6 and iter_num > 20:
            # 停滞状态下额外加速
            lr_scale *= min(1.5, 1.0 + self.stagnation_counter / 10.0)
            
        # 基于迭代阶段的调整
        phase_scale = 1.0
        if iter_num < 20:
            phase_scale = 0.8  # 初始阶段保守
        elif iter_num > 100:
            phase_scale = 1.2  # 后期阶段适当激进
            
        return base_lr * lr_scale * phase_scale

    def _dynamic_momentum_adjust(self, iter_num, grad_norm, recent_progress):
        """
        动态动量调整
        :param iter_num: 当前迭代次数
        :param grad_norm: 当前梯度范数
        :param recent_progress: 最近的优化进展
        :return: 当前迭代的动量因子
        """
        base_momentum = self.config['momentum']
        
        if not self.config['adaptive_momentum']:
            return base_momentum
            
        # 检测边界震荡(多次在边界附近反弹)
        boundary_oscillation = False
        if len(self.history['params']) > 5:
            recent_params = np.array(self.history['params'][-5:])
            for i in range(self.n_dim):
                lb, ub = self.PARAM_BOUNDS[i]
                # 检查是否有参数在边界附近多次变化方向
                at_lb = np.abs(recent_params[:, i] - lb) < 1e-4
                at_ub = np.abs(recent_params[:, i] - ub) < 1e-4
                if np.sum(at_lb) > 2 or np.sum(at_ub) > 2:
                    boundary_oscillation = True
                    break
        
        # 长时间停滞检测
        if recent_progress < 1e-7 and iter_num - self.last_decay_iter > 10:
            # 触发动量衰减
            momentum = base_momentum * self.config['momentum_decay']
            self.last_decay_iter = iter_num
        # 边界震荡检测
        elif boundary_oscillation:
            # 显著降低动量以减少边界震荡
            momentum = base_momentum * 0.5
        # 梯度急剧变化检测
        elif len(self.history['gradients']) > 1:
            prev_grad = self.history['gradients'][-1]
            grad_angle = np.sum(prev_grad * self.gradient(self.params)) / (np.linalg.norm(prev_grad) * grad_norm + 1e-12)
            if grad_angle < -0.3:  # 梯度方向显著变化
                momentum = base_momentum * 0.7  # 减小动量
            elif grad_angle > 0.9:  # 梯度方向稳定
                momentum = min(0.98, base_momentum * 1.05)  # 略微增加动量
            else:
                momentum = base_momentum
        else:
            momentum = base_momentum
            
        # 确保动量在合理范围内
        return np.clip(momentum, 0.5, 0.98)

    def _wall_rebound(self, params, velocity):
        """
        边界弹性反弹机制
        :param params: 当前参数(已投影到可行域)
        :param velocity: 当前动量速度
        :return: 调整后的速度
        """
        rebounded_velocity = velocity.copy()
        
        # 对每个维度检查边界碰撞
        for i in range(self.n_dim):
            lb, ub = self.PARAM_BOUNDS[i]
            
            # 下边界碰撞检测
            if abs(params[i] - lb) < 1e-6 and velocity[i] < 0:
                # 速度反向并衰减
                rebounded_velocity[i] = -velocity[i] * self.config['wall_rebound']
                
            # 上边界碰撞检测
            if abs(params[i] - ub) < 1e-6 and velocity[i] > 0:
                # 速度反向并衰减
                rebounded_velocity[i] = -velocity[i] * self.config['wall_rebound']
                
        return rebounded_velocity

    def _perform_line_search(self, params, direction, grad):
        """
        执行回溯线搜索确定最优步长
        :param params: 当前参数点
        :param direction: 搜索方向
        :param grad: 当前梯度
        :return: 最优步长
        """
        if not self.config['line_search']:
            return 1.0
            
        # 定义线搜索的目标函数
        def obj_func(alpha):
            new_params = self._project_params(params + alpha * direction)
            return self.objective(new_params)
            
        # 初始步长
        alpha0 = 1.0
        
        # 回溯线搜索
        c1 = 1e-4  # Armijo条件系数
        tau = 0.5   # 回溯因子
        alpha = alpha0
        f0 = self.objective(params)
        df0 = np.sum(grad * direction)  # 方向导数
        
        # 最多10次回溯
        for _ in range(10):
            new_params = self._project_params(params + alpha * direction)
            f_new = self.objective(new_params)
            
            # 检查Armijo条件
            if f_new <= f0 + c1 * alpha * df0:
                return alpha
                
            # 回溯步长
            alpha *= tau
            
        # 如果回溯失败,返回保守步长
        return 0.01

    def _check_convergence(self):
        """增强型多维收敛检测"""
        if len(self.history['deltas']) < 5:
            return False

        # 收集最近的收敛指标
        recent_deltas = self.history['deltas'][-5:]
        recent_gradients = [np.linalg.norm(g) for g in self.history['gradients'][-5:]]
        recent_costs = self.history['costs'][-5:]
        
        # 计算5种收敛指标
        criteria = {
            'param_change': np.mean(recent_deltas) < self.config['epsilon'],
            'gradient_norm': np.mean(recent_gradients) < self.config['epsilon'],
            'cost_change': np.mean([abs(recent_costs[i+1] - recent_costs[i]) for i in range(4)]) < self.config['epsilon']**2,
            'cost_plateau': np.std(recent_costs) < self.config['epsilon']**2,
            'acceleration': np.std(recent_deltas) < self.config['epsilon']**2
        }
        
        # 更新收敛历史
        self.converge_history = np.roll(self.converge_history, -1, axis=1)
        self.converge_history[:, -1] = [float(v) for v in criteria.values()]
        
        # 检查持续稳定收敛(5种指标在最近3次迭代都满足)
        stable_convergence = np.all(self.converge_history[:, -3:], axis=1)
        
        return np.sum(stable_convergence) >= 3  # 至少3种指标达到稳定收敛

    def _update_best_solution(self):
        """更新历史最优解"""
        current_cost = self.objective(self.params)
        if current_cost < self.best_cost:
            self.best_cost = current_cost
            self.best_params = self.params.copy()
            return True
        return False

    def optimize(self, verbose=True):
        """执行增强科学优化流程"""
        self.params = self._project_params(self.params)
        current_cost = self.objective(self.params)
        self.best_cost = current_cost
        self.best_params = self.params.copy()
        
        # 初始化历史记录
        self.history['params'].append(self.params.copy())
        self.history['costs'].append(current_cost)
        raw_grad = self.gradient(self.params)
        self.history['gradients'].append(raw_grad)
        self.history['violations'].append(0.0)
        self.history['deltas'].append(0.0)
        self.history['learning_rates'].append(self.config['base_lr'])
        self.history['momentum_factors'].append(self.config['momentum'])
        
        try:
            for iter_num in range(self.config['max_iters']):
                # 1. 计算原始梯度
                raw_grad = self.gradient(self.params)
                grad_norm = np.linalg.norm(raw_grad)
                
                # 计算最近的优化进展
                recent_progress = 0.0
                if len(self.history['costs']) > 5:
                    recent_costs = self.history['costs'][-5:]
                    recent_progress = abs(recent_costs[0] - recent_costs[-1])
                
                # 2. 动态梯度调整
                adjusted_grad = self._dynamic_gradient_adjust(self.params, raw_grad)
                
                # 3. 梯度投影(确保搜索方向的可行性)
                projected_grad = self._project_gradient(self.params, adjusted_grad)
                proj_grad_norm = np.linalg.norm(projected_grad)
                
                # 4. 自适应动量调整
                momentum_factor = self._dynamic_momentum_adjust(iter_num, grad_norm, recent_progress)
                
                # 5. 更新速度向量(动量法)
                search_direction = -projected_grad + momentum_factor * self.velocity
                
                # 6. 自适应学习率
                base_lr = self._adaptive_learning_rate(iter_num, proj_grad_norm, recent_progress)
                
                # 7. 线搜索确定最优步长
                step_size = self._perform_line_search(self.params, search_direction, projected_grad) * base_lr
                
                # 8. 参数更新
                new_params = self.params + step_size * search_direction
                
                # 9. 参数投影
                new_params = self._project_params(new_params)
                
                # 10. 计算实际更新量
                param_delta = np.linalg.norm(new_params - self.params)
                
                # 11. 边界反弹机制(调整速度/动量)
                self.velocity = search_direction  # 保存当前搜索方向作为速度
                self.velocity = self._wall_rebound(new_params, self.velocity)
                
                # 记录优化状态
                self.history['params'].append(new_params.copy())
                new_cost = self.objective(new_params)
                self.history['costs'].append(new_cost)
                self.history['gradients'].append(projected_grad.copy())
                self.history['violations'].append(np.linalg.norm(new_params - self._project_params(new_params)))
                self.history['deltas'].append(param_delta)
                self.history['learning_rates'].append(base_lr)
                self.history['momentum_factors'].append(momentum_factor)
                
                # 更新最优解
                is_improved = self._update_best_solution()
                
                # 更新停滞计数器
                if param_delta < self.config['epsilon'] / 10:
                    self.stagnation_counter += 1
                else:
                    self.stagnation_counter = 0
                
                # 检查收敛
                if self._check_convergence() and iter_num > 10:
                    if verbose:
                        print(f"\n✅ 严格收敛于第 {iter_num+1} 次迭代")
                        self._print_convergence_report()
                    return True
                
                # 更新当前参数
                self.params = new_params
                
                # 每10次迭代打印进度
                if verbose and (iter_num+1) % 10 == 0:
                    improve_mark = "*" if is_improved else " "
                    print(f"Iter {iter_num+1:4d} | Cost: {new_cost:.3e}{improve_mark} | "
                          f"Grad: {proj_grad_norm:.1e} | Δ: {param_delta:.1e} | "
                          f"LR: {base_lr:.2e} | M: {momentum_factor:.2f}")

            if verbose:
                print("\n⚠️ 达到最大迭代次数")
            
            # 恢复历史最优解
            self.params = self.best_params
            return False
        except Exception as e:
            if verbose:
                print(f"\n❌ 优化异常: {str(e)}")
            return False

    def _print_convergence_report(self):
        """打印增强型收敛诊断报告"""
        print("\n收敛诊断报告:")
        print("="*40)
        
        # 计算最终收敛指标
        final_cost = self.history['costs'][-1]
        best_cost = min(self.history['costs'])
        final_grad_norm = np.linalg.norm(self.history['gradients'][-1])
        final_delta = self.history['deltas'][-1]
        
        # 四个主要收敛指标
        criteria = {
            '参数变化量': final_delta,
            '梯度范数': final_grad_norm,
            '成本值': final_cost,
            '最优成本差': abs(final_cost - best_cost),
            '边界违规量': self.history['violations'][-1]
        }
        
        for name, value in criteria.items():
            threshold = self.config['epsilon'] if name != '成本值' else 0
            status = "✔️" if value < self.config['epsilon'] or name == '成本值' else "❌"
            print(f"{name:8s}: {value:.2e} {status}")
            
        # 收敛稳定性分析
        print("\n收敛稳定性分析:")
        print("-"*40)
        
        # 检查最近10次迭代的变化趋势
        if len(self.history['costs']) >= 10:
            recent_costs = self.history['costs'][-10:]
            recent_grads = [np.linalg.norm(g) for g in self.history['gradients'][-10:]]
            recent_deltas = self.history['deltas'][-10:]
            
            cost_trend = np.polyfit(np.arange(10), recent_costs, 1)[0]
            grad_trend = np.polyfit(np.arange(10), recent_grads, 1)[0]
            delta_trend = np.polyfit(np.arange(10), recent_deltas, 1)[0]
            
            trends = {
                "成本趋势": cost_trend,
                "梯度趋势": grad_trend,
                "变化趋势": delta_trend
            }
            
            for name, value in trends.items():
                trend_str = "持平" if abs(value) < 1e-6 else ("下降" if value < 0 else "上升")
                print(f"{name}: {trend_str} ({value:.2e})")

    def render_contour(self, resolution=100, margin=0.1, show_trajectory=True, highlight_best=True):
        """
        绘制增强型目标函数等高线与优化轨迹
        :param resolution: 网格分辨率
        :param margin: 可视化边界扩展比例
        :param show_trajectory: 是否显示优化轨迹
        :param highlight_best: 是否高亮最优点
        """
        # 确定可视化范围(带边距)
        x_min, x_max = -margin, 1.0 + margin
        y_min, y_max = -margin, 1.0 + margin
        
        # 创建网格
        x = np.linspace(x_min, x_max, resolution)
        y = np.linspace(y_min, y_max, resolution)
        X, Y = np.meshgrid(x, y)
        
        # 计算每个网格点的函数值
        Z = np.zeros_like(X)
        for i in range(resolution):
            for j in range(resolution):
                Z[i, j] = self.objective(np.array([X[i, j], Y[i, j]]))
        
        # 创建图表
        plt.figure(figsize=(10, 8))
        
        # 绘制等高线
        levels = np.logspace(0, 5, 20)  # 对数尺度等高线
        contour = plt.contourf(X, Y, Z, levels=levels, cmap='viridis', alpha=0.7)
        plt.colorbar(contour, label='目标函数值 (对数尺度)')
        
        # 绘制可行域边界
        plt.plot([0, 1, 1, 0, 0], [0, 0, 1, 1, 0], 'r--', linewidth=2, label='可行域边界')
        
        # 绘制优化轨迹
        if show_trajectory and len(self.history['params']) > 1:
            trajectory = np.array(self.history['params'])
            plt.plot(trajectory[:, 0], trajectory[:, 1], 'b.-', linewidth=1, markersize=3, label='优化轨迹')
            plt.scatter(trajectory[0, 0], trajectory[0, 1], c='red', s=100, marker='o', label='初始点')
            plt.scatter(trajectory[-1, 0], trajectory[-1, 1], c='lime', s=100, marker='*', label='终点')
        
        # 高亮标记最优点
        if highlight_best and self.best_params is not None:
            plt.scatter(self.best_params[0], self.best_params[1], c='yellow', s=150, 
                       marker='*', edgecolors='black', linewidth=1, label='最优解')
        
        # 设置图表属性
        plt.xlim(x_min, x_max)
        plt.ylim(y_min, y_max)
        plt.xlabel('发电效率系数', fontsize=12)
        plt.ylabel('余热回收系数', fontsize=12)
        plt.title(f'冷热电联供优化 ({self.mode}): 目标函数等高线与优化轨迹', fontsize=14)
        plt.grid(True, linestyle='--', alpha=0.3)
        plt.legend(loc='upper right')
        
        return plt.gcf()  # 返回当前图形对象

    def visualize(self, save_path=None):
        """生成增强型多维可视化"""
        fig = plt.figure(figsize=(16, 12))
        
        # 创建2x2网格布局
        gs = fig.add_gridspec(2, 2)
        
        # 1. 等高线与优化轨迹
        ax1 = fig.add_subplot(gs[0, 0])
        
        # 定义可视化范围 (略微扩展可行域)
        margin = 0.05
        x_min, x_max = -margin, 1.0 + margin
        y_min, y_max = -margin, 1.0 + margin
        
        # 创建网格
        resolution = 100
        x = np.linspace(x_min, x_max, resolution)
        y = np.linspace(y_min, y_max, resolution)
        X, Y = np.meshgrid(x, y)
        
        # 计算每个网格点的函数值
        Z = np.zeros_like(X)
        for i in range(resolution):
            for j in range(resolution):
                Z[i, j] = self.objective(np.array([X[i, j], Y[i, j]]))
        
        # 绘制等高线
        levels = np.logspace(0, 5, 15)  # 对数尺度等高线
        contour = ax1.contourf(X, Y, Z, levels=levels, cmap='viridis', alpha=0.7)
        fig.colorbar(contour, ax=ax1, label='目标函数值 (对数尺度)')
        
        # 绘制可行域边界
        ax1.plot([0, 1, 1, 0, 0], [0, 0, 1, 1, 0], 'r--', linewidth=2, label='可行域边界')
        
        # 绘制优化轨迹
        if len(self.history['params']) > 1:
            trajectory = np.array(self.history['params'])
            ax1.plot(trajectory[:, 0], trajectory[:, 1], 'b.-', linewidth=1, markersize=3, label='优化轨迹')
            ax1.scatter(trajectory[0, 0], trajectory[0, 1], c='red', s=100, marker='o', label='初始点')
            ax1.scatter(trajectory[-1, 0], trajectory[-1, 1], c='lime', s=100, marker='*', label='终点')
            
            # 高亮标记最优点
            if self.best_params is not None:
                ax1.scatter(self.best_params[0], self.best_params[1], c='yellow', s=150, 
                           marker='*', edgecolors='black', linewidth=1, label='最优解')
        
        # 设置图表属性
        ax1.set_xlim(x_min, x_max)
        ax1.set_ylim(y_min, y_max)
        ax1.set_xlabel('发电效率系数', fontsize=12)
        ax1.set_ylabel('余热回收系数', fontsize=12)
        ax1.set_title('目标函数等高线与优化轨迹', fontsize=12)
        ax1.grid(True, linestyle='--', alpha=0.3)
        ax1.legend(loc='upper right', fontsize=10)
        
        # 2. 成本收敛曲线
        ax2 = fig.add_subplot(gs[0, 1])
        costs = np.array(self.history['costs'])
        iterations = np.arange(len(costs))
        
        # 对数尺度绘制成本
        ax2.semilogy(iterations, costs, 'r-', linewidth=2, label='成本值')
        
        # 标记最优成本
        best_idx = np.argmin(costs)
        ax2.scatter(best_idx, costs[best_idx], c='yellow', s=100, marker='*', 
                   edgecolors='black', linewidth=1, label=f'最优成本: {costs[best_idx]:.2e}')
        
        # 设置图表属性
        ax2.set_xlabel('迭代次数', fontsize=12)
        ax2.set_ylabel('综合成本 (对数尺度)', fontsize=12)
        ax2.set_title('成本收敛过程', fontsize=12)
        ax2.grid(True, alpha=0.3)
        ax2.legend(loc='upper right', fontsize=10)
        
        # 3. 参数变化过程
        ax3 = fig.add_subplot(gs[1, 0])
        params = np.array(self.history['params'])
        
        ax3.plot(iterations, params[:, 0], 'g-', linewidth=2, label='发电效率')
        ax3.plot(iterations, params[:, 1], 'm-', linewidth=2, label='余热回收')
        
        # 设置图表属性
        ax3.set_xlabel('迭代次数', fontsize=12)
        ax3.set_ylabel('参数值', fontsize=12)
        ax3.set_title('参数迭代过程', fontsize=12)
        ax3.set_ylim(-0.05, 1.05)  # 略微扩展可行域
        ax3.grid(True, alpha=0.3)
        ax3.legend(loc='best', fontsize=10)
        
        # 4. 优化诊断指标
        ax4 = fig.add_subplot(gs[1, 1])
        
        # 绘制梯度范数
        grads = np.array([np.linalg.norm(g) for g in self.history['gradients']])
        ax4.semilogy(iterations, grads, 'k-', linewidth=2, label='梯度范数')
        
        # 绘制参数变化量
        deltas = np.array(self.history['deltas'])
        ax4.semilogy(iterations, deltas, 'b--', linewidth=2, label='参数变化量')
        
        # 绘制学习率
        if len(self.history['learning_rates']) == len(iterations):
            lrs = np.array(self.history['learning_rates'])
            ax4_twin = ax4.twinx()
            ax4_twin.plot(iterations, lrs, 'r-.', linewidth=1.5, label='学习率')
            ax4_twin.set_ylabel('学习率', fontsize=12, color='r')
            ax4_twin.tick_params(axis='y', colors='r')
        
        # 设置图表属性
        ax4.set_xlabel('迭代次数', fontsize=12)
        ax4.set_ylabel('优化指标 (对数尺度)', fontsize=12)
        ax4.set_title('优化诊断指标', fontsize=12)
        ax4.grid(True, alpha=0.3)
        ax4.legend(loc='upper left', fontsize=10)
        if 'ax4_twin' in locals():
            ax4_twin.legend(loc='upper right', fontsize=10)
        
        # 调整布局
        plt.tight_layout()
        
        # 保存图表
        if save_path:
            plt.savefig(save_path, dpi=150, bbox_inches='tight')
        
        return fig
        
# 交互界面函数
def get_valid_input(prompt, value_type, default=None, validation=None):
    """获取并验证用户输入"""
    while True:
        try:
            user_input = input(prompt).strip()
            if not user_input and default is not None:
                return default
            value = value_type(user_input)
            if validation and not validation(value):
                raise ValueError
            return value
        except:
            print(f"输入无效,请重新输入(默认值:{default})")

def run_optimization_with_ui():
    """运行带交互界面的优化程序"""
    print("\n" + "="*60)
    print("     冷热电联供系统科学优化平台 v5.0 (科学增强版)")
    print("="*60)
    
    # 模式选择
    mode_choice = get_valid_input(
        "选择运行模式 (1:高电负荷 2:热电平衡 3:余热优先): ",
        str, '2', lambda x: x in ['1','2','3']
    )
    mode_map = {'1':'mode1', '2':'mode2', '3':'mode3'}
    selected_mode = mode_map[mode_choice]
    
    # 初始参数配置
    if get_valid_input("是否自定义初始参数? (y/n): ", str, 'n').lower() == 'y':
        x1 = get_valid_input("发电效率初始值 (0-1): ", float, 0.0, lambda x: 0<=x<=1)
        x2 = get_valid_input("余热回收初始值 (0-1): ", float, 0.0, lambda x: 0<=x<=1)
        initial_params = [x1, x2]
    else:
        # 为不同模式提供更好的默认初始值
        if selected_mode == 'mode1':
            initial_params = [0.5, 0.5]
        elif selected_mode == 'mode2':
            initial_params = [0.3, 0.3]
        else:
            initial_params = [0.2, 0.6]
    
    # 高级优化配置
    print("\n高级优化参数配置:")
    print("-"*40)
    
    config = {
        'max_iters': get_valid_input(
            "最大迭代次数 (默认1000): ", int, 1000, lambda x: x>0),
        'epsilon': get_valid_input(
            "收敛阈值 (默认1e-5): ", float, 1e-5, lambda x: x>0),
        'base_lr': get_valid_input(
            "基础学习率 (默认0.1): ", float, 0.1, lambda x: x>0),
        'momentum': get_valid_input(
            "动量系数 (默认0.9): ", float, 0.9, lambda x: 0<=x<1)
    }
    
    # 高级配置(可选)
    if get_valid_input("是否配置高级参数? (y/n): ", str, 'n').lower() == 'y':
        config.update({
            'line_search': get_valid_input(
                "启用线搜索 (y/n): ", lambda x: x.lower() == 'y', True),
            'adaptive_momentum': get_valid_input(
                "启用自适应动量 (y/n): ", lambda x: x.lower() == 'y', True),
            'wall_rebound': get_valid_input(
                "边界反弹系数 (0-1): ", float, 0.5, lambda x: 0<=x<=1),
            'gradient_boost': get_valid_input(
                "梯度增强系数 (1-5): ", float, 2.0, lambda x: 1<=x<=5)
        })
    
    print("\n" + "-"*60)
    print("开始优化计算...")
    
    # 初始化优化器
    optimizer = ScientificCCHPOptimizer(
        mode=selected_mode,
        config=config
    )
    optimizer.params = np.array(initial_params)
    
    # 执行优化
    start_time = time.time()
    success = optimizer.optimize(verbose=True)
    elapsed = time.time() - start_time
    
    # 显示结果
    print("\n优化结果报告")
    print("="*60)
    print(f"运行模式: \t{optimizer.mode}")
    print(f"初始参数: \t{initial_params}")
    print(f"迭代次数: \t{len(optimizer.history['params'])-1}")
    print(f"计算耗时: \t{elapsed:.3f}秒")
    print(f"最终发电效率: \t{optimizer.params[0]:.6f}")
    print(f"最终余热回收: \t{optimizer.params[1]:.6f}")
    print(f"最终成本值: \t{optimizer.objective(optimizer.params):.6e}")
    
    if optimizer.best_params is not None and not np.array_equal(optimizer.params, optimizer.best_params):
        print("\n最优解记录:")
        print(f"最优发电效率: \t{optimizer.best_params[0]:.6f}")
        print(f"最优余热回收: \t{optimizer.best_params[1]:.6f}")
        print(f"最优成本值: \t{optimizer.best_cost:.6e}")
    
    # 可视化
    if get_valid_input("是否生成可视化图表? (y/n): ", str, 'y').lower() == 'y':
        save_dir = "fig5"
        if not os.path.exists(save_dir):
            os.makedirs(save_dir)
        save_path = os.path.join(save_dir, f"cchp_optimization_{selected_mode}.png")
        optimizer.visualize(save_path)
        print(f"图表已保存至 {os.path.abspath(save_path)}")
    
    return optimizer

if __name__ == "__main__":
    run_optimization_with_ui()
\end{python}
\end{document}